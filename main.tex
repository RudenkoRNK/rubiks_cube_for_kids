\documentclass[a4paper,12pt]{article}
% 13pt size font
\usepackage[fontsize=12pt]{scrextend}

%=============================== Packages ===============================

% Encodings
\usepackage[T1,T2A]{fontenc}
\usepackage{fontspec}
\usepackage[english,russian]{babel}


% NewDocumentCommand and conditional expressions
\usepackage{xparse}
\usepackage{iftex}
\usepackage{etoolbox}

\usepackage[usenames,dvipsnames,svgnames,table,rgb]{xcolor}

\usepackage{enumitem}

\usepackage{multicol}

% Margins
\usepackage{geometry}

% Centered sections
\usepackage{sectsty}

% More intelligent hyphenation
\usepackage{microtype}

% Tables
\usepackage{array,booktabs}
\usepackage{tabularx,tabulary}
\usepackage{longtable}
\usepackage{multirow}

% Pictures
\usepackage{graphicx}
\usepackage{wrapfig}
\usepackage{epstopdf}
\usepackage{adjustbox}

% Bibliography
\usepackage[backend=biber,
    bibencoding=utf8,
    sorting=nty,
    maxcitenames=2,
    style=numeric-comp]
{biblatex}

% Search in PDF document
\usepackage{cmap}

\usepackage{hyperref}
\usepackage[font=small,figurename=Рисунок]{caption}
\usepackage{subcaption}

% Correct quotation
\usepackage{fvextra}
\usepackage{csquotes}

% First line indent
\usepackage{indentfirst}
\usepackage{parskip}

% Symbols like copyright, degree
\usepackage{textcomp}

% Bold, italics, etc.
\usepackage{soul}

% Number of pages in document
\usepackage{lastpage}

% Local style setup
\usepackage{afterpage}

% Header and footer
\usepackage{fancyhdr}

% Linespacing
\usepackage{setspace}

% Russian letters in formulas
\usepackage{mathtext}

\usepackage{amsmath}

% Extended math symbol collection
\usepackage{amssymb}

\usepackage{amsfonts}

\usepackage{amsthm}

% Fix bugs in ams
\usepackage{mathtools}

% Math font with \mathscr command
\usepackage{mathrsfs}

% Additional math symbols
\usepackage{stmaryrd}

% Code highlighting
\usepackage[outputdir=build]{minted}

% A comprehensive SI units package
\usepackage{siunitx}

% \nicefrac \unitfrac and \unit command
\usepackage{units}

% \qty, \norm, \abs, \div, \grad, \curl, \laplacian, etc.
\usepackage{physics}

% Graphics
\usepackage{tikz}
\usetikzlibrary{matrix}
\usepackage{pgfplots}
\pgfplotsset{compat=1.17}
\usepackage{pgfplotstable}

\usepackage{rubikcube,rubikrotation,rubikpatterns}

%=============================== Customizations ===============================
\defaultfontfeatures{Renderer=Basic,Ligatures={TeX}}
\setmainfont{CMU Serif}
\setsansfont{CMU Sans Serif}
\setmonofont{CMU Typewriter Text}

% Set linespacing
%\singlespacing
%\onehalfspacing
%\doublespacing

% Set up margins
\geometry{top=20mm}
\geometry{bottom=20mm}
\geometry{left=20mm}
\geometry{right=20mm}

% Set up headers
\pagestyle{fancy}
\renewcommand{\headrulewidth}{0mm}
\lfoot{}
\rfoot{}
\rhead{}
\chead{}
\lhead{}
%\cfoot{Нижний в центре}

\graphicspath{{images/}{build/images/}{build/}}

% Picture box margins
\setlength\fboxsep{3pt}
\setlength\fboxrule{1pt}

\newcolumntype{x}[1]{>{\centering\let\newline\\\arraybackslash\hspace{0pt}}p{#1}}

% First line indent
\setlength{\parindent}{0cm}
\frenchspacing

\hypersetup{
    unicode=true,
    pdftitle={Кубик Рубика для детей},
    pdfauthor={Никита Руденко},
    % False means refs in boxes, true means color refs
    colorlinks=true,
    linkcolor=red,
    citecolor=blue,
    filecolor=magenta,
    urlcolor=blue
}

\captionsetup[figure]{
    labelformat=simple,
    labelsep=endash,
    justification=centering
}

\captionsetup[listing]{
    labelformat=simple,
    labelsep=endash,
    justification=centering
}

\SetupFloatingEnvironment{listing}{name=Листинг}

% Center sections
%\allsectionsfont{\centering}

% Only show numbers if formula is referenced
%\mathtoolsset{showonlyrefs=true}

% Display formula number on the left
%\usepackage{leqno}


\addbibresource{bibliography.bib}

%=============================== Support commands ===============================
\makeatletter
\NewDocumentCommand{\swapcommands}{m m}{%
    \let\@swapcommands #1%
    \let #1 #2%
    \let #2\@swapcommands%
}
\NewDocumentCommand{\wrapensuremath}{m}{%
    % Throw error, if "\N#1" is already defined.
    \expandafter\@ifdefinable\csname #1@ensuremath\endcsname{%
        % Save old meaning
        \expandafter
        \let\csname #1@ensuremath\expandafter\endcsname
        \csname #1\endcsname
        % Define new macro
        \expandafter\edef\csname #1\endcsname{%
            \noexpand\ensuremath{%
                \expandafter\noexpand\csname #1@ensuremath\endcsname
            }%
        }%
    }%
}

% Parse curly or round brackets
\NewDocumentCommand{\@PCRB}{m m}{
    \IfNoValueTF{#1}{\IfNoValueTF{#2}{}{\quantity(#2)}}
    {#1 \IfNoValueTF{#2}{}{(#2)}}
}
% Parse math operator
\NewDocumentCommand{\parsemathoperator}{m m m}{
    \ensuremath{#1\@PCRB{#2}{#3}}
}




%=============================== Russian typography ===============================
\NewDocumentCommand{\varle}{}{\leqslant}
\NewDocumentCommand{\varge}{}{\geqslant}
\swapcommands{\epsilon}{\varepsilon}
\swapcommands{\phi}{\varphi}
\swapcommands{\kappa}{\varkappa}
\swapcommands{\emptyset}{\varnothing}
\swapcommands{\le}{\varle}
\swapcommands{\ge}{\varge}

%=============================== Ensure math for common symbols ===============================
\wrapensuremath{alpha}
\wrapensuremath{beta}
\wrapensuremath{gamma}   %\wrapensuremath{Gamma}
\wrapensuremath{delta}   %\wrapensuremath{Delta}
\wrapensuremath{epsilon}
\wrapensuremath{zeta}
\wrapensuremath{eta}
\wrapensuremath{theta}   %\wrapensuremath{Theta}
\wrapensuremath{iota}
\wrapensuremath{kappa}
\wrapensuremath{lambda}  %\wrapensuremath{Lambda}
\wrapensuremath{mu}
\wrapensuremath{nu}
\wrapensuremath{xi}      %\wrapensuremath{Xi}
\wrapensuremath{pi}      %\wrapensuremath{Pi}
\wrapensuremath{rho}
\wrapensuremath{sigma}   %\wrapensuremath{Sigma}
\wrapensuremath{tau}
\wrapensuremath{upsilon} %\wrapensuremath{Upsilon}
\wrapensuremath{phi}     %\wrapensuremath{Phi}
\wrapensuremath{chi}
\wrapensuremath{psi}     %\wrapensuremath{Psi}
\wrapensuremath{omega}   %\wrapensuremath{Omega}
\wrapensuremath{iff}
\wrapensuremath{rightarrow}
\wrapensuremath{leftarrow}
\wrapensuremath{implies}
\wrapensuremath{impliedby}

%=============================== Common math ===============================
\NewDocumentCommand{\aqty}{m}{\ensuremath{\left<#1\right>}}
\NewDocumentCommand{\n}{m}{\ensuremath{\centernot{#1}}}
%\NewDocumentCommand{\x}{}{\ensuremath{\times}}
\NewDocumentCommand{\ox}{}{\ensuremath{\otimes}}
\NewDocumentCommand{\transpose}{m}{#1^\top}
\NewDocumentCommand{\T}{}{^\top}
\NewDocumentCommand{\orthogonal}{}{^\top}
\NewDocumentCommand{\ort}{}{^\top}
\NewDocumentCommand{\herm}{}{^\dagger}
\NewDocumentCommand{\suchthat}{}{\colon}
\NewDocumentCommand{\func}{m m m}{\ensuremath{#1\colon #2\rightarrow #3}} % Function: X -> Y
\NewDocumentCommand{\isomorph}{}{\ensuremath{\cong}}
\NewDocumentCommand{\divby}{}{\mathrel{\rotatebox{90}{\ensuremath{\hskip-1pt.{}.{}.}}}}
\NewDocumentCommand{\ord}{g d()}{\parsemathoperator{\operatorname{ord}}{#1}{#2}}
\NewDocumentCommand{\sgn}{g d()}{\parsemathoperator{\operatorname{sgn}}{#1}{#2}}
\NewDocumentCommand{\Id }{g d()}{\parsemathoperator{\operatorname{Id}}{#1}{#2}}

%=============================== Sets ===============================
\NewDocumentCommand{\set}{m m}{\ensuremath{\qty{#1\colon #2}}}
\NewDocumentCommand{\powerset}{g d()}{\parsemathoperator{\mathcal{P}}{#1}{#2}}
\NewDocumentCommand{\permset}{g d()}{\parsemathoperator{\mathcal{S}}{#1}{#2}}
\NewDocumentCommand{\N}{}{\ensuremath{\mathbb{N}}}
\NewDocumentCommand{\Z}{}{\ensuremath{\mathbb{Z}}}
\NewDocumentCommand{\Q}{}{\ensuremath{\mathbb{Q}}}
\NewDocumentCommand{\R}{}{\ensuremath{\mathbb{R}}}
\RenewDocumentCommand{\C}{}{\ensuremath{\mathbb{C}}} % was U+030F "COMBINING DOUBLE GRAVE ACCENT".
\NewDocumentCommand{\union}{}{\ensuremath{\cup}}
\NewDocumentCommand{\intxn}{}{\ensuremath{\cap}}
\NewDocumentCommand{\Union}{}{\ensuremath{\bigcup}}
\NewDocumentCommand{\Intxn}{}{\ensuremath{\bigcap}}

%=============================== Mathematical logic ===============================
\NewDocumentCommand{\imply}{}{\ensuremath{\:\rightarrow\:}}
\NewDocumentCommand{\implied}{}{\ensuremath{\:\leftarrow\:}}
\NewDocumentCommand{\eqdef}{}{\ensuremath{\:\longleftrightarrow\:}}
\swapcommands{\equiv}{\eqdef}
\NewDocumentCommand{\iffdef}{}{\ensuremath{\aqty{\eqdef}}} % iff by definition
\NewDocumentCommand{\eqsym}{}{\ensuremath{\eqcirc}} % Equivalent symbol by symbol
\NewDocumentCommand{\A}{m g d()}{\parsemathoperator{\forall #1\:}{#2}{#3}}
\NewDocumentCommand{\E}{m g d()}{\parsemathoperator{\exists #1\:}{#2}{#3}}
\RenewDocumentCommand{\a}{}{\ensuremath{\land}}
\RenewDocumentCommand{\o}{}{\ensuremath{\lor}}   % was little emptyset


%=============================== Group theory ===============================
\RenewDocumentCommand{\ker}{g d()}{\parsemathoperator{\operatorname{ker}}{#1}{#2}} % improvement of previous ker
\NewDocumentCommand{\imag}{g d()}{\parsemathoperator{\operatorname{im}}{#1}{#2}}
\NewDocumentCommand{\subgr}{}{\ensuremath{\leqslant}}
\NewDocumentCommand{\rsubgr}{}{\ensuremath{\geqslant}}
\NewDocumentCommand{\normsubgr}{}{\ensuremath{\trianglelefteqslant}}
\NewDocumentCommand{\rnormsubgr}{}{\ensuremath{\trianglerighteqslant}}
\NewDocumentCommand{\grcenter}{g d()}{\parsemathoperator{\operatorname{Z}}{#1}{#2}}
\NewDocumentCommand{\centralizer}{m g d()}{\parsemathoperator{\operatorname{C}_{#1}}{#2}{#3}}
\NewDocumentCommand{\normalizer}{m g d()}{\parsemathoperator{\operatorname{N}_{#1}}{#2}{#3}}
\NewDocumentCommand{\factorgr}{m m}{\ensuremath{\nicefrac{#1}{#2}}}
\NewDocumentCommand{\rfactorgr}{m m}{%
    \reflectbox{%
        \nicefrac{\reflectbox{\ensuremath{#1}}}%
        {\reflectbox{\ensuremath{#2}}}}%
}




%=============================== Theorems ===============================
\newtheoremstyle{plainbreak}
{}%                                  % Space above, empty = `usual value'
{}%                                  % Space below
{\rmfamily}%                         % Body font
{}%                                  % Indent amount (empty = no indent, \parindent = para indent)
{\bfseries}%                         % Thm head font
{}%                                  % Punctuation after thm head
{\newline}%                          % Space after thm head: \newline = linebreak
{}%                                  % Thm head spec

\theoremstyle{plainbreak}
% Name in code, name in text, counter dependency
\newtheorem{theorem}{Теорема}[section]
% Counter here is placed in between, which means that it is not dependent but the same counter
\newtheorem{proposition}[theorem]{Утверждение}
\newtheorem{corollary}{Следствие}[theorem]
\newtheorem{lemma}[theorem]{Лемма}
\newtheorem*{example}{Пример}
\newtheorem*{examples}{Примеры}
\newtheorem{problem}{Задача}[section]
\newtheorem{exercise}{Упражение}[section]
\newtheorem{definition}{Определение}[section]
\newtheorem{denotation}[definition]{Обозначение}

\theoremstyle{remark}
\newtheorem*{remark}{Замечание}
\newtheorem*{solution}{Решение}
\newtheorem*{answer}{Ответ}


\newcommand*\@proofenvname{proof}
\newcommand*{\theoremlistshack}{%
    \leavevmode
    \ifx\@currenvir\@proofenvname
    \else
        \vspace{-\baselineskip}
    \fi
    \par%
    \everypar{\setbox\z@\lastbox\everypar{}}%
}

\makeatother


%\def\ANSWERS

\author{Никита Руденко}
\title{Кубик Рубика для детей}
\date{}


\renewcommand{\RubikCubeSolved}{\RubikSolvedConfig{G}{B}{Y}{W}{R}{O}}

\begin{document}
\maketitle

\section{Введение}

Кубик Рубика — механическая головоломка, изобретённая в 1974 году венгерским скульптором и преподавателем архитектуры Эрнё Рубиком \cite{wiki:cube}.

\begin{figure}[h]
    \centering
    \RubikCubeSolved
    \ShowCube{\textwidth/2}{1}{\DrawRubikCubeRU}
    \caption{Собранный кубик Рубика $3\times 3\times 3$}
    \label{fig:solved}
\end{figure}

У кубика есть несколько \emph{сторон} или, по-другому, \emph{граней}.
На рисунке \ref{fig:solved} представлен полностью собранный кубик, у которого видно три стороны --- желтую, красную и зеленую.

\begin{exercise}
    Внимательно рассмотрите кубик.
    Сколько всего у него сторон?
    Сколько всего цветов?
    Какие это цвета?
\end{exercise}

\ifdef{\ANSWERS}{
    \begin{answer}
        У кубика 6 граней и 6 цветов.
    \end{answer}
}{}


\begin{exercise}
    Кубик как-будто состоит из кубиков поменьше, называемых \emph{элементами}.
    Посчитайте, из какого количества элементов состоит кубик Рубика размера $3\times 3\times 3$.
\end{exercise}

\ifdef{\ANSWERS}{
    \begin{answer}
        Он состоит из 26 или 27 элементов, в зависимости от того будете ли вы учитывать элемент в самом центре кубика, которого не видно.
    \end{answer}
}{}

Кубик состоит из трех типов элементов, называемых \emph{центрами}, \emph{ребрами} и \emph{уголками}.
На рисунке \ref{fig:elements} приведены пояснения.

\begin{figure}[H]
    \centering
    \begin{tabular}{ccc}
        \RubikCubeGreyAll
        \RubikFaceUp
        {X}{X}{X}
        {X}{Y}{X}
        {X}{X}{X}
        \RubikFaceRight
        {X}{X}{X}
        {X}{G}{X}
        {X}{X}{X}
        \RubikFaceFront
        {X}{X}{X}
        {X}{R}{X}
        {X}{X}{X}
        \ShowCube{0.3\textwidth}{1}{\DrawRubikCubeRU}
         &
        \RubikCubeGreyAll
        \RubikFaceUp
        {X}{Y}{X}
        {Y}{X}{Y}
        {X}{Y}{X}
        \RubikFaceRight
        {X}{G}{X}
        {G}{X}{G}
        {X}{G}{X}
        \RubikFaceFront
        {X}{R}{X}
        {R}{X}{R}
        {X}{R}{X}
        \ShowCube{0.3\textwidth}{1}{\DrawRubikCubeRU}
         &
        \RubikCubeGreyAll
        \RubikFaceUp
        {Y}{X}{Y}
        {X}{X}{X}
        {Y}{X}{Y}
        \RubikFaceRight
        {G}{X}{G}
        {X}{X}{X}
        {G}{X}{G}
        \RubikFaceFront
        {R}{X}{R}
        {X}{X}{X}
        {R}{X}{R}
        \ShowCube{0.3\textwidth}{1}{\DrawRubikCubeRU}
        \\
        Центральные элементы
         &
        Ребра
         &
        Уголки
    \end{tabular}
    \caption{Три типа элементов кубика.}
    \label{fig:elements}
\end{figure}

\begin{exercise}
    Перемешайте кубик.
    Найдите центральный элемент белого цвета.
    Какого цвета центральный элемент с противоположной стороны кубика?
    Повторите то же самое с зеленым и оранжевым центральными элементами.

    Снова хорошо перемешайте кубик и ответьте на те же вопросы.
    Изменились ли ответы?
    Если нет, то попробуйте перемешать еще раз и снова ответить на те же вопросы.
\end{exercise}
\ifdef{\ANSWERS}{
    \begin{answer}
        Белый цвет всегда располагается напротив желтого, красный напротив оранжевого, а зеленый напротив синего.
        Математики говорят, что это \emph{инвариант кубика}.
    \end{answer}
}{}

\begin{theorem}[Первый инвариант кубика]\label{th:inv_1}
    Никакими разрешенными поворотами кубика невозможно изменить относительное положение центральных элементов.
    Белый всегда будет располагаться напротив желтого, красный напротив оранжевого, а зеленый напротив синего.
\end{theorem}
\ifdef{\ANSWERS}{
    \begin{proof}
        Нужно заметить, что любой поворот кубика может менять положение уголков и ребер, но никогда не меняет положения центральных элементов.
    \end{proof}
}{}
\begin{exercise}
    Докажите теорему \ref{th:inv_1}.
\end{exercise}

\begin{corollary}\label{th:adjacent}
    Не существует ребер бело-желтого цвета или уголков, на которых есть одновременно зеленый и синий цвета.
\end{corollary}
\ifdef{\ANSWERS}{
    \begin{proof}
        Рассмотрим полностью собранный кубик.
        Если бы было ребро бело-желтого цвета, то белый и желтый должны были быть смежными цветами.
        Значит, желтый и белый центральные элементы тоже должны были быть смежными.
        По теореме \ref{th:inv_1} это невозможно.
    \end{proof}
}{}
\begin{exercise}
    Докажите следствие \ref{th:adjacent}.
\end{exercise}

Поскольку положение центральных элементов не может меняться, то цвет стороны собранного кубика всегда определяется его центральным элементом.
Поэтому, даже на несобранном кубике говорят <<желтая сторона>> и имеют ввиду ту сторону, у которой цвет центрального элемента желтый.


\section{Формулы}
Кубик собирается с помощью \emph{алгоритма}, который описывается \emph{формулами}.
Для того чтобы научиться собирать кубик необходимо уметь читать и записывать эти формулы.

Возьмите кубик двумя руками.
Сторона которую вы видите прямо перед собой называется \emph{фронтальной}.
Она обозначается буквой \emph{Ф}.

Сторона, которую вы держите правой рукой называется \emph{правой}, обозначается буквой \emph{П}.
Сторона, которую вы держите левой рукой называется \emph{левой}, обозначается буквой \emph{Л}.
Аналогично, сверху находится \emph{верхняя (В)}, снизу --- \emph{нижняя (Н)} стороны.

\begin{exercise}
    Переведите на английский язык названия сторон: верх, низ, лево, право, фронт, зад.
\end{exercise}
\ifdef{\ANSWERS}{
    \begin{answer}
        Up, Down, Left, Right, Front, Back.
    \end{answer}
}{}

На рисунке \ref{fig:sides} показаны названия сторон.
Обратите внимание, что названия сторон не связаны с цветами.
Например, желтый цвет может быть верхней, так и нижней стороной, в зависимости от того как ориентирован ваш кубик.

\begin{figure}[H]
    \centering
    \RubikCubeSolved
    \begin{tabular}{x{0.45\textwidth} x{0.45\textwidth}}
        \ShowCube{0.3\textwidth}{1}{
            \DrawRubikCubeLD
            \node[align=left] at (-2,1) {\large Лево};
            \node[align=left] at (1,-1.5) {\large Низ};
            \node[align=left] at (1.5,1.5) [white]{\large Фронт};
        }
         &
        \ShowCube{0.3\textwidth}{1}{
            \DrawRubikCubeRU
            \node[align=left] at (5,2) {\large Право};
            \node[align=left] at (2,4.5) {\large Верх};
            \node[align=left] at (1.5,1.5) [white]{\large Фронт};
        }
    \end{tabular}
    \caption{
        Обозначения сторон.
        Верхняя (В), Правая (П), Левая (Л), Нижняя (Н) и Фронтальная (Ф).
    }
    \label{fig:sides}
\end{figure}

В таблице \ref{tab:rotations} приведены изображения поворотов и их обозначений.


\begin{table}[H]
    \centering
    \RubikCubeSolved
    \begin{tabular}{x{0.3\textwidth} x{0.3\textwidth} x{0.3\textwidth}}
        \ShowCube{0.3\textwidth}{1}{
            \DrawRubikCubeRU
            \draw[->,line width=1mm] (2.5, 0.5) -- (2.5,2.5);
        }
         &
        \ShowCube{0.3\textwidth}{1}{
            \DrawRubikCubeRU
            \draw[->,line width=1mm] (2.5, 2.5) -- (0.5,2.5);
        }
         &
        \ShowCube{0.3\textwidth}{1}{
            \DrawRubikCubeRU
            \draw[->,line width=1mm] (2.5,1.5) arc (0:-340:1);
        }
        \\
        П --- поворот правой стороны от себя
         &
        В --- поворот верхней стороны справа налево
         &
        Ф --- поворот фронтальной стороны по часовой стрелке
        \\ & \\

        \ShowCube{0.3\textwidth}{1}{
            \DrawRubikCubeRU
            \draw[->,line width=1mm] (0.5,2.5) -- (0.5, 0.5);
        }
         &
        \ShowCube{0.3\textwidth}{1}{
            \DrawRubikCubeRU
            \draw[->,line width=1mm] (0.5, 0.5) -- (2.5, 0.5);
        }
         &
        \ShowCube{0.3\textwidth}{1}{
            \DrawRubikCubeRU
            \draw[->,line width=0.5mm] (3.85, 1.2) -- (3.85,3.2);
            \draw[->,line width=0.5mm] (3.2, 3.85) -- (1.2,3.85);
        }
        \\
        Л --- поворот левой стороны на себя
         &
        Н --- поворот нижней стороны слева направо
         &
        З --- поворот задней стороны против часовой стрелки
    \end{tabular}
    \caption{Повороты кубика и их обозначения}
    \label{tab:rotations}
\end{table}

Добавления штриха (') к букве означает поворот в обратную сторону.
Добавление цифры означает повторение поворота несколько раз.

\begin{exercise}
    Что происходит с кубиком при выполнении формулы П4?
\end{exercise}
\ifdef{\ANSWERS}{
    \begin{answer}
        Ничего.
        Математики говорят, что это \emph{тождественное} или \emph{единичное} преобразование.
    \end{answer}
}{}

\section{Первый слой}
Кубик $3\times 3\times 3$ состоит из трех слоев.
На рисунке \ref{fig:slices} показаны эти слои.

\begin{figure}[H]
    \centering
    \begin{tabular}{x{0.3\textwidth} x{0.3\textwidth} x{0.3\textwidth}}
        \RubikCubeGreyAll
        \RubikFaceDownAll{W}
        \RubikSliceBottomR{R}{R}{R} {G}{G}{G}
        \ShowCube{0.3\textwidth}{1}{\DrawRubikCubeRD}
         &
        \RubikCubeGreyAll
        \RubikSliceMiddleR{R}{R}{R} {G}{G}{G}
        \ShowCube{0.3\textwidth}{1}{\DrawRubikCubeRU}
         &
        \RubikCubeGreyAll
        \RubikFaceUpAll{Y}
        \RubikSliceTopR{R}{R}{R} {G}{G}{G}
        \ShowCube{0.3\textwidth}{1}{\DrawRubikCubeRU}
        \\
        Первый слой
         &
        Второй слой
         &
        Третий слой
    \end{tabular}
    \caption{Три слоя кубика $3\times 3\times 3$.}
    \label{fig:slices}
\end{figure}

Для сборки кубика необходимо сначала собрать первый слой, затем второй слой и, наконец, третий слой.
Сборка первого слоя начинается со сборки \emph{креста}.

Пример собранного креста изображен на рисунке \ref{fig:cross} слева.
\begin{figure}[H]
    \centering
    \begin{tabular}{x{0.45\textwidth} x{0.45\textwidth}}
        \RubikCubeGreyAll
        \RubikFaceDown
        {X}{W}{X}
        {W}{W}{W}
        {X}{W}{X}
        \RubikSliceBottomR{X}{R}{X} {X}{O}{X}
        \ShowCube{\textwidth/2}{1}{\DrawRubikCubeRD}
         &
        \RubikCubeGreyAll
        \RubikFaceDown
        {X}{W}{X}
        {W}{W}{W}
        {X}{W}{X}
        \RubikSliceMiddleR{X}{R}{X} {X}{G}{X}
        \RubikSliceBottomR{X}{R}{X} {X}{G}{X}
        \ShowCube{\textwidth/2}{1}{\DrawRubikCubeRD}
        \\
        Крест
         &
        Правильный крест
    \end{tabular}
    \caption{Примеры собранных крестов.}
    \label{fig:cross}
\end{figure}

\begin{exercise}
    Научитесь собирать крест.
\end{exercise}

Второй шаг заключается в сборке \emph{правильного креста}.
У правильного креста цвета ребер совпадают с цветами центральных элементов, как показано на рисунке \ref{fig:cross} справа.


\begin{exercise}
    Может ли число ребер креста, которые совпадают цветом с центральными элементами равняться 3?
    Докажите это.
\end{exercise}
\ifdef{\ANSWERS}{
    \begin{answer}
        Не может.
        Без потери общности будем считать, что собран белый крест и совпадают ребра зеленого, синего и красного цветов.
        Оставшееся ребро не может быть бело-зеленым, бело-синим и бело-красным, так как они уже использованы.
        Но ребер с белым цветом всего четыре, поэтому остается только ребро бело-оранжевого цвета.
        При этом оставшаяся сторона тоже оранжевого цвета.
    \end{answer}
}{}

\begin{exercise}
    Соберите крест.
    Повращайте слой с крестом.
    Постарайтесь добиться наибольшего совпадения цветов ребер с цветом центральных элементов.
    Повторите упражнение несколько раз.
\end{exercise}

\begin{theorem}\label{th:cross}
    Поворотами слоя с собранным крестом можно всегда добиться совпадения цветов как минимум двух ребер с центральными элементами.
\end{theorem}
\iffalse
    \ifdef{\ANSWERS}{
        \begin{proof}
            Довернем слой с крестом так, чтобы совпадал хотя бы одного ребра с цветом центра (очевидно, что так сделать можно).
            Возможно два варианта: либо после поворота уже совпадает минимум два ребра, либо совпадает ровно одно.
            В первом случае теорема выполнена.
            Рассмотрим второй случай.
        \end{proof}
    }{}
\fi

\begin{exercise}
    Попробуйте доказать теорему \ref{th:cross}.
\end{exercise}


Предположим, что цвета ровно двух ребер креста совпадают с центральными элементами.
Существуют два принципиально разных случая совпадения этих ребер.
В первом случае не совпадают два соседних ребра.
Во втором --- два противоположных.
В таблице \ref{tab:cross} приведены эти примеры и формулы для их решения.
Не забудьте ориентировать ваш кубик также, как на картинках.

\begin{table}[H]
    \centering
    \begin{tabular}{x{0.45\textwidth} x{0.45\textwidth}}
        \RubikCubeGreyAll
        \RubikFaceUp
        {X}{W}{X}
        {W}{W}{W}
        {X}{W}{X}
        \RubikSliceMiddleR{X}{G}{X} {X}{R}{X}
        \RubikSliceTopR{X}{R}{X} {X}{G}{X}
        \ShowCube{0.3\textwidth}{1}{\DrawRubikCubeRU}
         &
        \RubikCubeGreyAll
        \RubikFaceUp
        {X}{W}{X}
        {W}{W}{W}
        {X}{W}{X}
        \RubikSliceMiddleR{X}{G}{X} {X}{R}{X}
        \RubikSliceTopR{X}{B}{X} {X}{R}{X}
        \ShowCube{0.3\textwidth}{1}{\DrawRubikCubeRU}
        \\
        Первый случай.
        Не совпадают два соседних ребра.
        Решение --- ПВ'П'ВП
         &
        Второй случай.
        Не совпадают противоположные ребра.
        Решение --- П2Л2В2П2Л2
    \end{tabular}
    \caption{Сборка правильного креста}
    \label{tab:cross}
\end{table}


\begin{exercise}
    Постарайтесь понять как работают формулы из таблицы \ref{tab:cross}.
\end{exercise}

После сборки креста необходимо собрать уголки первого слоя.
Ориентируйте кубик так, чтобы собранный крест располагался снизу.
Найдите уголок с белым цветом на верхнем слое и поставьте его так, чтобы он находился между боковыми сторонами других двух своих цветов.
Поставьте уголок на нижний слой, воспользовавшись формулами из таблицы \ref{tab:first_layer}.
\begin{table}[H]
    \centering
    \begin{tabular}{x{0.3\textwidth} x{0.3\textwidth} x{0.3\textwidth}}
        \RubikCubeGreyAll
        \RubikFaceUp
        {X}{X}{X}
        {X}{X}{X}
        {X}{X}{G}
        \RubikSliceTopR{X}{X}{R} {W}{X}{X}
        \RubikSliceMiddleR{X}{R}{X} {X}{G}{X}
        \RubikSliceBottomR{X}{R}{X} {X}{G}{X}
        \ShowCube{0.3\textwidth}{1}{\DrawRubikCubeRU}
         &
        \RubikCubeGreyAll
        \RubikFaceUp
        {X}{X}{X}
        {X}{X}{X}
        {X}{X}{R}
        \RubikSliceTopR{X}{X}{W} {G}{X}{X}
        \RubikSliceMiddleR{X}{R}{X} {X}{G}{X}
        \RubikSliceBottomR{X}{R}{X} {X}{G}{X}
        \ShowCube{0.3\textwidth}{1}{\DrawRubikCubeRU}
         &
        \RubikCubeGreyAll
        \RubikFaceUp
        {X}{X}{X}
        {X}{X}{X}
        {X}{X}{W}
        \RubikSliceTopR{X}{X}{G} {R}{X}{X}
        \RubikSliceMiddleR{X}{R}{X} {X}{G}{X}
        \RubikSliceBottomR{X}{R}{X} {X}{G}{X}
        \ShowCube{0.3\textwidth}{1}{\DrawRubikCubeRU}
        \\
        Первый случай.
        Уголок смотрит вправо.
        Решение --- ПВП'
         &
        Второй случай.
        Уголок смотрит влево.
        Решение --- ВПВ'П'
         &
        Третий случай.
        Уголок смотрит вверх.
        Решение --- ПВ2П'В', после чего случай сводится к предыдущим двум.
    \end{tabular}
    \caption{Сборка уголков первого слоя.}
    \label{tab:first_layer}
\end{table}

\pagebreak
\printbibliography

\end{document}
